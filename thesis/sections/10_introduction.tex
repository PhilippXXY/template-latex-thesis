%% ----------------
%% | Introduction |
%% ----------------
\chapter{Introduction}
\label{ch:Introduction}

\glsresetall

\section{Abbreviations}
\label{sec:Introduction:Abbreviations}
This thesis uses abbreviations from a \texttt{.bib} file via \texttt{@abbreviation} entries.
For example: \gls{a:ecu}. If a sentence starts with a glossary entry, use \verb|\Gls| instead of \verb|\gls|.
\Glspl{a:ecu} are common in embedded systems.

You can also call the abbreviations directly like this:
\begin{itemize}
    \item long (lowercase): \abbrLongLC{a:ecu}
    \item long (Title Case): \abbrLongTC{a:ecu}
\end{itemize}

\section{Symbol}
\label{sec:Introduction:Symbol}
A symbol example: expected bytes per polling interval is \sym{s:bytes_event_per_ecu} with unit \symUnit{s:bytes_event_per_ecu}.

\section{Citation}
\label{sec:Introduction:Citation}
IEEE citation example: Shannon's paper \cite{shannon1948}.
